% Options for packages loaded elsewhere
\PassOptionsToPackage{unicode}{hyperref}
\PassOptionsToPackage{hyphens}{url}
%
\documentclass[
]{article}
\title{Final Project Brief}
\author{}
\date{\vspace{-2.5em}}

\usepackage{amsmath,amssymb}
\usepackage{lmodern}
\usepackage{iftex}
\ifPDFTeX
  \usepackage[T1]{fontenc}
  \usepackage[utf8]{inputenc}
  \usepackage{textcomp} % provide euro and other symbols
\else % if luatex or xetex
  \usepackage{unicode-math}
  \defaultfontfeatures{Scale=MatchLowercase}
  \defaultfontfeatures[\rmfamily]{Ligatures=TeX,Scale=1}
\fi
% Use upquote if available, for straight quotes in verbatim environments
\IfFileExists{upquote.sty}{\usepackage{upquote}}{}
\IfFileExists{microtype.sty}{% use microtype if available
  \usepackage[]{microtype}
  \UseMicrotypeSet[protrusion]{basicmath} % disable protrusion for tt fonts
}{}
\makeatletter
\@ifundefined{KOMAClassName}{% if non-KOMA class
  \IfFileExists{parskip.sty}{%
    \usepackage{parskip}
  }{% else
    \setlength{\parindent}{0pt}
    \setlength{\parskip}{6pt plus 2pt minus 1pt}}
}{% if KOMA class
  \KOMAoptions{parskip=half}}
\makeatother
\usepackage{xcolor}
\IfFileExists{xurl.sty}{\usepackage{xurl}}{} % add URL line breaks if available
\IfFileExists{bookmark.sty}{\usepackage{bookmark}}{\usepackage{hyperref}}
\hypersetup{
  pdftitle={Final Project Brief},
  hidelinks,
  pdfcreator={LaTeX via pandoc}}
\urlstyle{same} % disable monospaced font for URLs
\usepackage[margin=1in]{geometry}
\usepackage{longtable,booktabs,array}
\usepackage{calc} % for calculating minipage widths
% Correct order of tables after \paragraph or \subparagraph
\usepackage{etoolbox}
\makeatletter
\patchcmd\longtable{\par}{\if@noskipsec\mbox{}\fi\par}{}{}
\makeatother
% Allow footnotes in longtable head/foot
\IfFileExists{footnotehyper.sty}{\usepackage{footnotehyper}}{\usepackage{footnote}}
\makesavenoteenv{longtable}
\usepackage{graphicx}
\makeatletter
\def\maxwidth{\ifdim\Gin@nat@width>\linewidth\linewidth\else\Gin@nat@width\fi}
\def\maxheight{\ifdim\Gin@nat@height>\textheight\textheight\else\Gin@nat@height\fi}
\makeatother
% Scale images if necessary, so that they will not overflow the page
% margins by default, and it is still possible to overwrite the defaults
% using explicit options in \includegraphics[width, height, ...]{}
\setkeys{Gin}{width=\maxwidth,height=\maxheight,keepaspectratio}
% Set default figure placement to htbp
\makeatletter
\def\fps@figure{htbp}
\makeatother
\setlength{\emergencystretch}{3em} % prevent overfull lines
\providecommand{\tightlist}{%
  \setlength{\itemsep}{0pt}\setlength{\parskip}{0pt}}
\setcounter{secnumdepth}{-\maxdimen} % remove section numbering
\ifLuaTeX
  \usepackage{selnolig}  % disable illegal ligatures
\fi

\begin{document}
\maketitle

\hypertarget{todays-learning-objectives}{%
\section{\texorpdfstring{Today's Learning
Objectives}{Today's Learning Objectives }}\label{todays-learning-objectives}}

\begin{itemize}
\tightlist
\item
  Understand what is required for the final project
\item
  Give an introduction to the requirements
\item
  Give an introduction to the expectations and delivery approach
\end{itemize}

\textbf{Duration - 1/2 hour}

\hypertarget{the-final-project}{%
\section{The Final Project}\label{the-final-project}}

\hypertarget{project-aims-and-learning-objectives}{%
\subsection{Project aims and learning
objectives}\label{project-aims-and-learning-objectives}}

Over the next week and a bit you will work individually to carry out an
\textbf{end-to-end data project}. This will allow you to consolidate and
expand everything you have learned on the course.

The projects vary in terms of the questions asked. They may require a
machine learning solution or it may be an in-depth analysis deep dive
into a specific dataset looking for key insights. All should include
\textbf{appropriate visualisation outputs}.

You will have to \textbf{present your findings using slides}. If you are
doing the PDA you will also need to document your analysis steps in the
form of a \textbf{report with embedded visualisations and text} (R
Markdown or Jupyter Notebook) detailing key steps and insights, with
appropriate documentation of data source, quality and ethics and any
assumptions you've made. You can organise your project folders however
you see fit, but remember to make your code reproducible, and keep the
project organised and easy for others to navigate.

You can do your final project in either \textbf{R or Python}.

Remember the objective of this project is not to necessarily have the
most impressive final presentation (you can always continue working on
it after the course!). Use this project as a chance to get more
\textbf{practice of skills}, whether that be coding Python or maybe you
want to use R to get the data wrangling and then want to focus on
practising modelling. It's up to you!

\hypertarget{the-brief}{%
\subsection{The Brief}\label{the-brief}}

Each project has its own brief contained within the relevant folder.

Consider the following learning objectives when trying to answer your
chosen brief:

\begin{itemize}
\tightlist
\item
  Capture requirements from the business and translate these into an
  achievable analysis project
\item
  Create a project plan to deliver an outcome
\item
  Identify the key insights and communicate these back to the business
\item
  Maintain required client confidentiality
\end{itemize}

\textbf{Please note:} These briefs were written by your instructors. The
data is openly available, and you will be able to post your project to a
public GitHub. While the companies themselves did not write the brief,
there is no issue with you showing them your work by tagging them on
Github/LinkedIn/Twitter.

\hypertarget{timescales}{%
\subsection{Timescales}\label{timescales}}

Below is the proposed plan of activities for the next week:

\begin{longtable}[]{@{}ll@{}}
\toprule
Date & Activity \\
\midrule
\endhead
Friday & Project handout \& starting a plan \\
Monday & Present your plan to an instructor \\
Tuesday & Project \\
Wednesday & Project \\
Thursday & 1-to-1 check in with instructors \\
Friday & Project \\
Weekend & Project \\
Monday & 1-to-1 check in with instructors \\
Tuesday & Project \\
Wednesday & Project presentations \\
Thursday & PDA \\
Friday & \textbf{Graduation!} 🎂🥂🎈🎉🎊 \\
\bottomrule
\end{longtable}

\hypertarget{expectations}{%
\subsection{Expectations}\label{expectations}}

As this is an end-to-end data project we'd like you to carry out and
document each stage of your project. The steps we'd like to see are:

\begin{enumerate}
\def\labelenumi{\arabic{enumi}.}
\tightlist
\item
  \textbf{Defining:} identifying the question being solved and why it is
  important to the business
\item
  \textbf{Planning:} plan the analysis and rough timings over the
  project period
\item
  \textbf{Understanding:} initial understanding of provided data and
  identify any data augmentation
\item
  \textbf{Analysing:} manipulation, modelling and validating
\item
  \textbf{Concluding:} inferring and communicating findings through
  reports and visualisations
\item
  \textbf{Implementing:} recommendations for real-world implementation,
  next steps and any limitations to using the findings and outputs.
\end{enumerate}

\hypertarget{confidentiality}{%
\subsection{Confidentiality}\label{confidentiality}}

Please consider whether there are any confidentiality or ethical issues
with the data in your project. The \textbf{current project briefs that
we provide all use open data} so it should not be a problem. However,
you should still consider whether there is any confidential or
identifiable information that should be anonymised, or whether there is
anything you should be wary of putting on a public GitHub repository. If
you are using your own real data from a client, there should be clear
rules set out on whether you are able to directly refer to the client
name and publish any of the data/results to GitHub. Do speak to an
instructor if you're unsure about any confidentiality issues with any of
the data you're using!

\hypertarget{final-presentation}{%
\subsection{Final Presentation}\label{final-presentation}}

In the presentation session you will have \textbf{up to 10 minutes},
plus time for questions to \textbf{present your findings and outputs}
back to the business.

We'd like you to:

\begin{itemize}
\tightlist
\item
  Talk through your approach to the business question
\item
  Discuss any challenges encountered and how you addressed these
\item
  Present a story of your final insights which compels your client to
  action
\item
  If your solution would require implementation, discuss how this could
  be done
\item
  Any enhancements you would pick up if you'd had more time
\item
  Any lessons for the future
\end{itemize}

\hypertarget{asking-questions-during-project-weeks}{%
\subsection{Asking questions during project
weeks}\label{asking-questions-during-project-weeks}}

Instructors will be available to answer questions during the project
week but there are requirements for coming with questions:

\begin{itemize}
\tightlist
\item
  Provide a background explanation to your question/task - we may not be
  as familiar with your project as you are.
\item
  Evidence you have tried researching a solution before coming to us -
  we will be asking what you've tried looking up first (whether that be
  on Google or the notes).\\
\item
  We will be aiming to provide guidance on your problem but not
  necessarily handing you an answer (because often there isn't a single
  `right' answer and it's up to you to make assumptions).
\end{itemize}

This is not us being cruel - after this project you won't have
instructors to hand to answer questions so \textbf{the more practice you
get problem solving the better}. Increased practice problem solving is a
much more valuable skill that a more `complete' final project but being
handed the answers.

\hypertarget{github}{%
\subsection{GitHub}\label{github}}

We expect you to have the project on a \textbf{new public Github repo},
to which you commit and push regularly any progress you're making on the
project. Once you've created your repo, \textbf{send the link to your
instructors} so we can check how you're getting on! Please also use the
\textbf{homework forms} to update us on how it's going and if you're
facing any issues.

\hypertarget{pda}{%
\subsection{PDA}\label{pda}}

If you are completing the PDA, please refer to the provided PDA
documents in the \texttt{final\_project\_pda\_info} folder
\textbf{before you get started on the project} to make sure you will be
able to meet all the additional requirements. You can also refer to the
all learning outcomes being targeted throughout your project work to
make sure you're doing the necessary things to pass. Please do ask
instructors if you need anything in the PDA documents clarified.

\end{document}
